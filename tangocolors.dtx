% \iffalse meta-comment
%
% Copyright (C) 2009 by Daniel G. Siegel <daniel@dgsiegel.net>
% Copyright (C) 2023 by Hilmar Preuße <hille42@web.de>
% ---------------------------------------------------------------------------
% This work may be distributed and/or modified under the
% conditions of the LaTeX Project Public License, either version 1.3
% of this license or (at your option) any later version.
% The latest version of this license is in
%   http://www.latex-project.org/lppl.txt
% and version 1.3 or later is part of all distributions of LaTeX
% version 2005/12/01 or later.
%
% This work has the LPPL maintenance status `maintained'.
%
% The Current Maintainer of this work is Daniel G. Siegel and
% Hilmar Preuße.
%
% This work consists of the files tangocolors.dtx and tangocolors.ins
% and the derived filebase tangocolors.sty.
%
% \fi
%
% \iffalse
%<*driver>
\ProvidesFile{tangocolors.dtx}
%</driver>
%<package>\NeedsTeXFormat{LaTeX2e}[1999/12/01]
%<package>\ProvidesPackage{tangocolors}
%<*package>
    [2023/03/20 0.2 Tango colors for LaTeX]
%</package>
%
%<*driver>
\documentclass{ltxdoc}
\usepackage{tangocolors}[2023/03/20]
\EnableCrossrefs
\CodelineIndex
\RecordChanges
\begin{document}
  \DocInput{tangocolors.dtx}
  \PrintChanges
  \PrintIndex
\end{document}
%</driver>
% \fi
%
% \CheckSum{89}
%
% \CharacterTable
%  {Upper-case    \A\B\C\D\E\F\G\H\I\J\K\L\M\N\O\P\Q\R\S\T\U\V\W\X\Y\Z
%   Lower-case    \a\b\c\d\e\f\g\h\i\j\k\l\m\n\o\p\q\r\s\t\u\v\w\x\y\z
%   Digits        \0\1\2\3\4\5\6\7\8\9
%   Exclamation   \!     Double quote  \"     Hash (number) \#
%   Dollar        \$     Percent       \%     Ampersand     \&
%   Acute accent  \'     Left paren    \(     Right paren   \)
%   Asterisk      \*     Plus          \+     Comma         \,
%   Minus         \-     Point         \.     Solidus       \/
%   Colon         \:     Semicolon     \;     Less than     \<
%   Equals        \=     Greater than  \>     Question mark \?
%   Commercial at \@     Left bracket  \[     Backslash     \\
%   Right bracket \]     Circumflex    \^     Underscore    \_
%   Grave accent  \`     Left brace    \{     Vertical bar  \|
%   Right brace   \}     Tilde         \~}
%
%
% \changes{0.1}{2009/05/11}{Initial release}
% \changes{0.2}{2023/03/20}{Converted to DTX file}
%
% \DoNotIndex{\newcommand,\newenvironment}
%
% \providecommand*{\url}{\texttt}
% \GetFileInfo{tangocolors.dtx}
% \title{The \textsf{tangocolors} package}
% \author{Daniel G. Siegel \url{daniel@dgsiegel.net} \\
% and \\ Hilmar Preuße \url{hille42@web.de}}
% \date{\fileversion~from \filedate}
%
% \maketitle
%
% \section{Introduction}
%
% This package allows to use colors from the Tango color
% palette\footnote{\url{http://tango.freedesktop.org/Tango\_Icon\_Theme\_Guidelines}}
% easily in \LaTeX. It may be distributed and/or modified
%
% \begin{enumerate}
% \item under the \LaTeX{} Project Public License and/or
% \item under the GNU Public License.
% \end{enumerate}
%
% \section{Usage}
%
% The Tango color palette defines some color names and their RGB codes.
% This \LaTeX{} macro package implements these color names, so
% one can easily access these colors by their names. The package
% uses the \verb|xcolor| package, so please refer to the documentation
% to this package to learn how to access these defined colors.
%
% \DescribeMacro{\dumptangocolors}
%
% A macro to dump a table of the available additional colors like this.
% \begin{verbatim}
% \documentclass[11pt,a4paper]{article}
% 
% \usepackage{tangocolors}
% 
% \begin{document}
%   \dumptangocolors
% \end{document}
% \end{verbatim}

% \dumptangocolors
%
% \StopEventually{}
%
% \section{Implementation}
%
% \iffalse
%<*package>
% \fi
%
%    \begin{macrocode}
\NeedsTeXFormat{LaTeX2e}
\ProvidesPackage{tangocolors}[2023/03/20 v0.2 Tango colors for LaTeX]
\PassOptionsToPackage{table}{xcolor}
\RequirePackage{xcolor}
\definecolor{butter1}{rgb}{0.988,0.914,0.310}
\definecolor{butter2}{rgb}{0.929,0.831,0.000}
\definecolor{butter3}{rgb}{0.769,0.627,0.000}
\definecolor{orange1}{rgb}{0.988,0.686,0.243}
\definecolor{orange2}{rgb}{0.961,0.475,0.000}
\definecolor{orange3}{rgb}{0.808,0.361,0.000}
\definecolor{chocolate1}{rgb}{0.914,0.725,0.431}
\definecolor{chocolate2}{rgb}{0.757,0.490,0.067}
\definecolor{chocolate3}{rgb}{0.561,0.349,0.008}
\definecolor{chameleon1}{rgb}{0.541,0.886,0.204}
\definecolor{chameleon2}{rgb}{0.451,0.824,0.086}
\definecolor{chameleon3}{rgb}{0.306,0.604,0.024}
\definecolor{skyblue1}{rgb}{0.447,0.624,0.812}
\definecolor{skyblue2}{rgb}{0.204,0.396,0.643}
\definecolor{skyblue3}{rgb}{0.125,0.290,0.529}
\definecolor{plum1}{rgb}{0.678,0.498,0.659}
\definecolor{plum2}{rgb}{0.459,0.314,0.482}
\definecolor{plum3}{rgb}{0.361,0.208,0.400}
\definecolor{scarletred1}{rgb}{0.937,0.161,0.161}
\definecolor{scarletred2}{rgb}{0.800,0.000,0.000}
\definecolor{scarletred3}{rgb}{0.643,0.000,0.000}
\definecolor{aluminium1}{rgb}{0.933,0.933,0.925}
\definecolor{aluminium2}{rgb}{0.827,0.843,0.812}
\definecolor{aluminium3}{rgb}{0.729,0.741,0.714}
\definecolor{aluminium4}{rgb}{0.533,0.541,0.522}
\definecolor{aluminium5}{rgb}{0.333,0.341,0.325}
\definecolor{aluminium6}{rgb}{0.180,0.204,0.212}
%    \end{macrocode}
%
% \begin{macro}{\dumptangocolors}
%    \begin{macrocode}
\newcommand{\dumptangocolors}{%
  \begingroup
    \renewcommand{\arraystretch}{1.5}
    \begin{tabular}{ccc}
    \cellcolor{butter1}butter1          & \cellcolor{butter2}butter2                          & \cellcolor{butter3}butter3 \\
    \cellcolor{orange1}orange1          & \cellcolor{orange2}orange2                          & \cellcolor{orange3}orange3 \\
    \cellcolor{chocolate1}chocolate1    & \cellcolor{chocolate2}chocolate2                    & \cellcolor{chocolate3}chocolate3 \\
    \cellcolor{chameleon1}chameleon1    & \cellcolor{chameleon2}chameleon2                    & \cellcolor{chameleon3}chameleon3 \\
    \cellcolor{skyblue1}skyblue1        & \cellcolor{skyblue2}skyblue2                        & \cellcolor{skyblue3}\textcolor{white}{skyblue3} \\
    \cellcolor{plum1}plum1              & \cellcolor{plum2}plum2                              & \cellcolor{plum3}\textcolor{white}{plum3} \\
    \cellcolor{scarletred1}scarletred1  & \cellcolor{scarletred2}scarletred2                  & \cellcolor{scarletred3}\textcolor{white}{scarletred3} \\
    \cellcolor{aluminium1}aluminium1    & \cellcolor{aluminium2}aluminium2                    & \cellcolor{aluminium3}aluminium3 \\
    \cellcolor{aluminium4}aluminium4    & \cellcolor{aluminium5}\textcolor{white}{aluminium5} & \cellcolor{aluminium6}\textcolor{white}{aluminium6} \\
    \end{tabular}
  \endgroup
}
%    \end{macrocode}
% \end{macro}
%

%
% \iffalse
%</package>
% \fi
%
% \Finale
\endinput
